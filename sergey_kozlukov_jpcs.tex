\documentclass[a4paper]{jpconf}
\usepackage{graphicx}
\usepackage{iopams}
% amsmath defined substack
\usepackage{amsmath}

\usepackage{todonotes}

\begin{document}
\title{The method of similar operators in the study of graph spectra}

\author{Sergey V Kozlukov}

\address{\emph{Voronezh State University}, 1 Universitetskaya Ploshad', Voronezh, RU 394036}

\ead{ithesaboteur@yandex.com}

\begin{abstract}
    The~method~of~similar~operators~\cite{baskakov1983methods,baskakov2014memory,baskakov2017method,baskakov2013completeness}
        is used to investigate spectral properties
        of a certain class of matrices in the context of graphs~\cite{van2003graphs,cvetkovic1980spectra}.
    Specifically, we consider matrices
        obtained as combinations
        of kronecker products~\cite{bellman-matrices-kron,XIANG2005210}
        and small-norm perturbations.
    The estimates of the spectrum and the eigenvectors
        of such matrices are given.
\end{abstract}

\section{Introduction}
Consider a~directed graph on \( N \) vertices.
Its adjacency matrix is defined as
    a matrix
    \( A = (a_{ij}) \)
    of the size \( N\times N \),
    in which \( a_{ij} \)
    is the number of edges
    from the vertex \( i \)
    to the vertex \( j \).
One can also consider generalized
    adjacency matrices~\cite{van2003graphs}
    which are defined as
    arbitrary linear combinations
    of matrices \( A \) (or its transpose \( A^\mathtt{T} \)),
    \( E \) (the identity matrix)
    and \( J_N \) (the all-ones matrix).
These matrices arise naturally
    in some stochastic models.
Spectral  properties of these matrices
    often play a vital role in such models.
For instance, the markovian random walk on a~graph
    yields the notion of eigenvector centrality
    in a~network~\cite{bonacich1972factoring}.
The transition matrix of such a random walk
    is generalized adjacency matrix of a graph.
The score of the vertex \( i \)
    is defined as the \( i \)'th coordinate
    of the dominating left eigenvector
    of the transition matrix.
The largest left eigenvalue of an irreducible column-stochastic matrix
    is \( 1 \) and the dominating left eigenvector
    defines the only stationary distribution
    of a random walk.
The PageRank~\cite{ilprints422} algorithm
    originally used by Google
    to compute the eigenvector centrality
    relies on the Power-Method.
Its speed of convergence depends on
    the ratio of the two largest absolute eigenvalues.
Stability of the stationary distribution
    is determined~\cite{meyer1994sensitivity}
    by the condition number
    which is bounded from below
    by a spectral gap --- the distance between
    the two largest eigenvalues
    of the transition matrix.
The method of estimation of almost-invariant sets
    proposed in~\cite{schwartz2006fluctuation}
    relies as well on spectral decompositions of such matrices.
In the Susceptible-Infective-Susceptible model
    a viral spread in a network
    is modeled as a markov process
    with \( 2^N \) states.
An asymptotic (endemic or epidemic) behaviour of this system
    is determined by the spectral radius (the absolute largest eigenvalue)
    of the adjacency matrix
    and the rates of curage and infection.
For more details and comprehensive description
    of the graph spectra theory
    refer to~\cite{cvetkovic1980spectra,godsil2013algebraic}.

\section{Methods and materials}

We use the abstract method of similar operators
    to derive estimates of eigenvalues and eigenvectors
    of certain types of such matrices.
This method originates from Friedrichs~\cite{friedrichs1965advanced}
    and was later developed in abstract setting
    by Baskakov~\cite{baskakov1983methods,baskakov2014memory,baskakov2017method,baskakov2013completeness}.
Here we will only state the required notation and theorems
    in a simplified form that accounts for finite-dimensionality
    of the problem.

Let \( \mathbb{K}\in \{ \mathbb{R}, \mathbb{C} \} \)
    be a field of either real or complex numbers.
We consider the vector space \( \mathbb{K}^n,\ n\in \mathbb{N} \)
    to be supplied with Euclidean structure:
    \[
        (x, y){=}\sum_{k=1}^n x_k\overline{y_k},
        \ x{=}(x_1,\ldots, x_n),
        \ y=(y_1,\ldots, y_n)
        \in \mathbb{K}^n
        \]
    and the \( \mathrm{L}_2 \) norm:
    \(
        \|x\|_2^2{=}(x,x).
        \)
We also consider a canonical basis \( e_1, \ldots, e_n \)
    in \( \mathbb{K}^n \) given by
    \( {(e_i)}_j = \delta_{ij},\ i,j=\overline{1,n} \)
    (\(\delta_{ij} \) is the Kronecker symbol).
When \( V_1, V_2 \) are normed vector spaces
    we denote by \( L(V_1, V_2) \)
    the space of bounded linear mappings
    from \( V_1 \) to \( V_2 \).
An algebra of bounded linear endomorphisms
    from a Banach space \( V \)
    into itself
    is denoted by \( L(V) = L(V, V) \).
It is a Banach algebra with the operator norm:
    \[
        \|A\|_{\mathrm{op}} =
        \sup_{\substack{\|x\|=1,\\ x\in V}} \|A x\|,\ A\in L(V).
        \]
Together with \( L(\mathbb{K}^n, \mathbb{K}^m) \)
    we consider its isomorphic space \( \mathbb{K}^{m{\times}n} \)
    of matrices of the size \( m{\times}n \)
    with entries from a field \( \mathbb{K} \).
The space \( \mathbb{K}^{n{\times}n}\sim L(\mathbb{K}^n) \)
    forms a Banach algebra
    when supplied with a submultiplicative norm
    \( \|\cdot\| \),
    e.g.: \( \|A\|_{\mathrm{op}} = \sup_{\|x\|_2=1,\ x\in \mathbb{K}^n} \|A x\|_2,\ \)
    \( \|A\|_{\mathrm{F}} = \sqrt{\sum_{i,j} |a_{ij}|^2},\ \)
    for a matrix 
    \( A{=}(a_{ij})\in\mathbb{K}^{n\times n} \).
Finally we are going to use the isomorphic spaces
    \( L(L(\mathbb{K}^n)) \) and \( L(\mathbb{K}^{n{\times}n}) \)
    with the operator norm.
We will follow {Krein \todo{cite smth}}
    and refer to elements of \( L(\mathbb{K}^{n{\times}n}) \)
    as ``transformers''.

The spectrum of a matrix \( A \)
    (the set of its eigenvalues)
    will be denoted as \( \sigma(A) \).
We call two matrices \( A_1, A_2 \) \emph{similar}
    if there is an invertible matrix \( U \)
    (called similarity matrix)
    such that \( A_1 U = U A_2 \).
Similar matrices share some spectral properties:
    they are isospectral (\( \sigma(A_1) = \sigma(A_2) \))
    and \( U \) maps eigenvectors of one to another's:
    \( A_2 x = \lambda x \implies A_1 U x = \lambda U x \).

The most important notion
    of the abstract method of similar operators
    is that of an \emph{admissible triple}.
For our specific purposes it suffices to say
    that \( (\mathbb{K}^n, J, \Gamma) \)
    forms an \emph{admissible triple}
    for a matrix \( A\in\mathbb{K}^{n{\times}n} \)
    if the following conditions are met:
\begin{itemize}
    \item \( J, \Gamma \in L(\mathbb{K}^{n{\times}n}) \)
        are transformers;
    \item \( J \) is a projection (\( J^2 = J \));
    \item  \( \Gamma \) satisfies the equations:
        \[
            A \Gamma X - (\Gamma X) A = X - JX,
        \]
        \[
            J\Gamma X = 0,\ X\in\mathbb{K}^{n{\times}n}.
        \]
\end{itemize}

The main theorem of the method
    may be then formulated as follows:

\textbf{Theorem.}
{\it
    Consider a matrix \( A - B \)
        with \( A, B \in \mathbb{K}^{n{\times}n} \).
    Suppose \( (\mathbb{K}^{n{\times}n}, J, \Gamma) \)
        is an admissible triple for a matrix \( A \)
        and the following inequality holds:
        \[
            \|B\|\|\Gamma\|_{\mathrm{op}} \leq \frac14.
        \]

    Then there exists such a matrix \( X^o\in\mathbb{K}^{n{\times}n} \)
        that
        \( A - B \) is similar to a matrix \( A - J X^o \);
        the similarity matrix is \( E + \Gamma X^o \);
        the following estimates are valid:
        \[
            \|X^o - B\| \leq 3 \left\|B\right\|,
        \]
        \[
            \operatorname{spr}(X^o) \leq \|X^o\| \leq 4 \left\|B\right\|,
        \]
        where \( \operatorname{spr}(X^o) \)
        is the spectral radius of \( X^o \) (the largest absolute eigenvalue).
    Such \( X^o \) can be found as the limit of a convergent sequence
        \( \left( \Phi^k(0);\ k\in\mathbb{N} \right) \)
        in a Banach algebra \( \mathbb{K}^{n{\times}n} \).
        Here \( \Phi \) is a nonlinear mapping given by
    \[
        \Phi(X) = B\Gamma X - (\Gamma X)J(B + B\Gamma X) + B
    \]
        and \( \Phi^k = \underbrace{\Phi\circ\cdots\circ\Phi}_{k\ \text{copies}} \)
        denotes composition.
}

\section{Results and discussion}

\subsection*{Almost-complete graph example.}

Consider a digraph defined by the following adjacency matrix:
\[
    A = J_N - B = \begin{pmatrix}1 & \cdots & 1 \\ \vdots & \ddots & \vdots \\ 1 & \cdots & 1\end{pmatrix} - B,
\]

Here \( J_N \) is the all-ones matrix.
The unity on the intersection
of \( i \)'th row and \( j \)'th column of \( B \)
corresponds to an edge from \( i \) to \( j \)
being absent in the graph.

One can easily find the minimal annihilating polynomial of \( J_N \)
    to be \( \lambda(\lambda - N) \).
This comes from the fact that \( J_N^2 = N J_N \).
Consequently the spectrum of \( J_N \) is
\[
    \sigma(J_N) = \left\{0, N\right\}.
\]
The only non-zero eigenvalue \( N \) has a corresponding eigenvector
\[
    h_N = \frac{1}{\sqrt{N}} \left(1, \ldots, 1\right)\in\mathbb{R}^N.
\]
The null-space of \( J_N \) is orthogonal to \( h_N \)
    and has an orthonormal basis of vectors:
\[
    h_k = \frac{1}{\sqrt{k(k+1)}} \left(\underbrace{1, \ldots, 1}_{k\ \text{copies}}, -k, 0, \ldots, 0\right),\ k=\overline{1, N-1}.
\]

We conclude that the adjacency matrix \( A = J_N - B \) of this graph
    is similar to a matrix \( \mathcal{A} - \mathcal{B} \)
    where \( A \) is a block matrix (in what follows throughout this subsection block sizes are the same and will be omitted):
    \[
        \mathcal{A} = \left(\begin{array}{c|c}
        N & \mathbf{0}_{1{\times}(N{-}1)} \\ \hline
            \mathbf{0}_{(N{-}1){\times}1} & \mathbf{0}_{(N{-}1){\times}(N{-}1)}
        \end{array}\right) \in \mathbb{R}^{N{\times}N}
    \]
    and \( \mathcal{B} \) is obtained with a similarity transform:
    \[
        \mathcal{B} = U^{-1} B U \in \mathbb{R}^{N{\times}N}.
    \]
The similarity matrix \( U \) is given by stacking column-eigenvectors:
    \[
        U = \operatorname{columns}(h_N, h_1, \ldots, h_{N-1}).
    \]

Following the general method, we shall construct an admissible triple.
Since \( \mathcal{A} \) is block-diagonal
    it is natural to set \( J \) with the following formula:
    \[
        JX =
        \left(\begin{array}{c|c}
            x_{11} & \mathbf{0} \\ \hline
            \mathbf{0} & X_{22}
        \end{array}\right)
    \]
    for all block matrices
    \[
        X =
        \left(\begin{array}{c|c}
            x_{11} & X_{12} \\ \hline
            X_{21} & X_{22}
        \end{array}\right)\in\mathbb{K}^{N{\times}N}.
    \]
Then \( A - JX \) is block-diagonal for any \( X \)
    and its spectrum is the union
    of its diagonal blocks' spectra:
    \( \sigma(A - JX) = \{N, \sigma(-X_{22}) \} \).

Now we are able to find the corresponding transformer \( \Gamma \).
Suppose it is defined by a formula
    \[ \Gamma X = \begin{pmatrix}
        \Gamma_{11}(X) & \Gamma_{12}(X) \\
        \Gamma_{21}(X) & \Gamma_{22}(X)
        \end{pmatrix}.
    \]
Then the equations
    \[
        \mathcal{A} \Gamma X - (\Gamma X)\mathcal{A} =
        N
        \begin{pmatrix}
          0 & \Gamma_{12}(X) \\
          -\Gamma_{21}(X) & 0
        \end{pmatrix} = X - JX,
     \]
and \( J\Gamma X = 0 \) yield the result
    \[
        \Gamma X = \frac{1}{N} \begin{pmatrix} 0 & X_{12} \\ -X_{21} & 0 \end{pmatrix},\ X\in\mathbb{K}^{N{\times}N}.
    \]

The last step is to estimate the norms and apply the theorem.
One can verify that \( \|\Gamma\|_{\mathrm{op}} = \frac1N \).
It is also apparent that \( \|\mathcal{B}\|_{\mathrm{op}} = \|B\|_{\mathrm{op}} \leq \|B\|_{mathrm{F}} \)
    since multiplication by orthogonal matrix\( U \) is an isometry in Euclidean space.
The Frobenius norm \( {\|B\|_{\mathrm{F}} = \sqrt{\sum_{ij} b_ij^2} = M} \)
    of \( B \)
    reduces to the square root of the number of absent edges.
A direct implication is the following

\textbf{Theorem.}
{\it
    Suppose the number of absent edges is
    \[ M < \frac{1}{16} N^2. \]
    Then the spectrum of the adjacency matrix \( A = J_N - B \)
        can be represented as disjoint union
    \[
        \sigma(A) = \{ N - x_{11}^o \} \cup \sigma_2.
    \]
    The dominating eigenvector of \( A \) is
    \[
        \hat{h}_N = U(E+\Gamma X^o) e_1 =
            h_N - \frac1N (X_{21,(1)}^o h_1 + \cdots + X_{21, (N{-}1)}^o h_{N{-}1}),
    \]
    where \( X_{21,(i)}^o,\ i=\overline{1,N{-}1} \) are the coordinates
    of the vector \( X_{21}^o \).
    Moreover \( \hat{h}_N\in\mathbb{R}^{N} \),
    \( x_{11}^o\in\mathbb{R} \) and \( \sigma_2\subset\mathbb{C} \)
    satisfy the following bounds:
    \[
        \|\hat{h}_N - h_N\|_2 \leq 4\frac{\sqrt{M}}{N},
    \]
    \[
        \lvert x_{11}^o \rvert \leq 4\sqrt{M},
    \]
    \[
        \sup_{\lambda\in\sigma_2} \lvert\lambda\rvert \leq 4\sqrt{M}.
    \]
}

\subsection*{A-tiled matrix example.}

Now let \( A\in\mathbb{K}^{m{\times}m} \)
    and consider the following (``A-tiled'') block-matrix
    \[
        \mathbb{A} =
        \begin{pmatrix}
            A & \cdots & A \\
            \vdots & \ddots & \vdots \\
            A & \cdots & A
        \end{pmatrix}
        \in\mathbb{K}^{{mn}{\times}{mn}}
    \]
    and a perturbed matrix
    \[
        \mathbb{A} - \mathbb{B},\ \mathbb{B}\in\mathbb{K}^{{mn}{\times}{mn}}.
    \]

\textbf{Lemma.}
{\it
    Suppose \( A \) is an invertible self-adjoint matrix.
    Then it has \( m \) orthonormal eigenvectors \( h_1, \ldots, h_m \)
    (\(\left\|h_i\right\|_2 = 1,\ i{=}\overline{1,m}\))
    with the corresponding eigenvalues
    \( \lambda_1, \ldots, \lambda_m \neq 0\).
    The spectrum of \( \mathbb{A} \) is
    \[
        \sigma(\mathbb{A}) = \{0\}\cup n\sigma(A) = \{0\} \cup \{n\lambda;\ \lambda\in\sigma(A) \}.
    \]
    Non-zero eigenvalues of \( \mathbb{A} \)
        have corresponding block eigenvectors:
    \[
        f_j = \frac{1}{\sqrt{n}} (h_j, \ldots, h_j)\in \mathbb{K}^{mn},\ j=\overline{1,m}.
    \]
    The null-space of \( \mathbb{A} \)
        is formed by the following orthonormal system of vectors:
    \[
        f_{j,k} = \frac{1}{\sqrt{k(k+1)}}
        (
        \underbrace{e_j, \ldots, e_j}_{k\ \text{copies}},
        -ke_j,
        0, \ldots, 0
        ) \in\mathbb{K}^{{mn}{\times}{mn}}
    \]

    The matrix \( \mathbb{A} \) is similar to a block-diagonal matrix
    \[
        \mathcal{A} =
        \left(\begin{array}{c|c}
            \operatorname{diag}(n\lambda_1,\ldots,n\lambda_m) & \mathbf{0} \\ \hline
            \mathbf{0} & \mathbf{0}
        \end{array}\right)\in\mathbb{K}^{{mn}{\times}{mn}}.
    \]
    The similarity transform matrix is
    \[
        U = \operatorname{columns}
        \left(f_1, \ldots, f_m, f_{1,1}, \ldots, f_{1,n{-1}}, \ldots, f_{m,n{-}1}\right).
    \]
\/
}

Throughout this subsection
    we will consider block matrices
    of the size \( {mn}{\times}{mn} \)
    in the form
    \[
    X =
        \left(\begin{array}{c|c}
            \begin{matrix}
                x_{11} & \cdots & x_{1m} \\
                \vdots & \ddots & \vdots \\
                x_{m1} & \cdots & x_{mm}
            \end{matrix} &
            \begin{matrix}
                x_{1,m+1} \\
                \vdots \\
                x_{m,m+1}
            \end{matrix} \\ \hline
            \begin{matrix}
                x_{m+1,1} &
                \cdots &
                x_{m+1,m}
            \end{matrix} &
            X_{m+1,m+1}
        \end{array}\right),
    \]
where
\( X_{ij}      {=} x_{ij},
 \ X_{m{+}1,j} {=} x_{m{+}1,j},
 \ X_{i,m{+}1} {=} x_{i,m{+}1} \in \mathbb{K} \)
for \( 1 \leq {i,j} \leq m \),
and
\( X_{m{+}1,m{+}1} \) is a block of the size \( {m(n{-}1){\times}m(n-1)} \).

Just like before we are going to investigate
    the spectral behaviour of \( \mathbb{A} \) under perturbations.
We begin with constructing an admissible triple.
A natural choice for \( J \) in this case is
\[
        J X =
        \left(\begin{array}{c|c}
            \begin{matrix}
                x_{11} &  & 0 \\
                 & \ddots &  \\
                0 &  & x_{mm}
            \end{matrix} &
            \begin{matrix}
                0 \\
                \vdots \\
                0
            \end{matrix} \\ \hline
            \begin{matrix}
                0 & \cdots & 0
            \end{matrix} &
            X_{m+1,m+1}
        \end{array}\right).
\]

\textbf{Lemma.}
{\it
Suppose \( A \) has a simple spectrum,
    i.e.\ its eigenvalues are pairwise distinct:
    \( \lambda_i\neq\lambda_j \) for all \( 1\leq i{\neq}j \leq m \).

    Then a tuple \( (\mathbb{K}^{{mn}{\times}{mn}}, J, \Gamma) \)
        forms an admissible triple if we define
    \[
        \Gamma X = 
            \frac1n \left(\begin{array}{c|c}
            \begin{matrix}
                0               & \gamma_{12}x_{12} & \cdots & \gamma_{1m}x_{1m} \\
                \gamma_{21}x_{21}  & 0              & \cdots & \gamma_{2m}x_{2m} \\
                \vdots          & \vdots         & \ddots & \vdots & \ \\
                \gamma_{m1}x_{m1}  & \gamma_{m2}x_{m2} & \cdots & 0
            \end{matrix} &
            \begin{matrix}
                \gamma_{1,m+1}x_{1,m+1} \\
                \gamma_{2,m+1}x_{2,m+1} \\
                \vdots \\
                \gamma_{m,m+1}x_{m,m+1}
            \end{matrix} \\ \hline
            \begin{matrix}
                \gamma_{m{+}1,1}x_{m{+}1,1} &
                \gamma_{m{+}1,2}x_{m{+}1,2} &
                \cdots &
                \gamma_{m{+}1,m}x_{m{+}1,m}
            \end{matrix} &
            \mathbf{0}
        \end{array}\right),
    \]
    \[
        \gamma_{ij} = \left\{
            \begin{aligned}
                & \frac{1}{\lambda_i - \lambda_j},\ 1\leq i{\neq}j \leq m{+}1,\\
                & 0,\ i=j
            \end{aligned}
            \right.
    \]
    and we use the convention:
    \[
        \lambda_{m{+}1} = 0.
    \]

    The operator norm of \( \Gamma \) is:
    \[
        \|\Gamma\|_{\mathrm{op}} =
        \frac1n
        \frac{1}{\min_{1\leq i{\neq}j \leq m{+}1}|\lambda_i - \lambda_j|} =
        \]
    \[
        = \frac1n
         \max\left\{
         \frac{1}{
             \min\limits_{1\leq i{\neq}j \leq m }{|\lambda_i - \lambda_j|}},
         \frac{1}{
             \min\limits_{1\leq j \leq m}{|\lambda_j|}}
         \right\}
        \]
\/}

\textbf{Theorem.}
{\it
Suppose \( A \) has simple spectrum and
\[
    \left\| \mathbb{B} \right\|_{\mathrm{op}}
        \leq 
        \frac{n}{4}
         \min\left\{
             \min\limits_{1\leq i{\neq}j \leq m }{|\lambda_i - \lambda_j|},
             \min\limits_{1\leq j \leq m}{|\lambda_j|}
         \right\}.
 \]

Then the spectrum of disturbed matrix \( \mathbb{A} - \mathbb{B} \) is
\[
    \sigma\left(\mathbb{A}\right) =
        \left\{
            n\lambda_1 - x_{11}^o, \ldots, n\lambda_m - x_{mm}^o
        \right\}
    \cup \sigma_{m{+}1}.
\]

The eigenvectors
    \( \hat{f}_j,\ \hat{f}_{j,k},\ j{=}\overline{1,m},\ k{=}\overline{1,n{-1}} \)
    of the matrix \( \mathbb{A}{-}\mathbb{B} \),
    the values \( x_{jj}^o,\ j{=}\overline{1,m} \)
    and the set \( \sigma_{m{+}1} \) have the following bounds:
\[
    \lvert x_{jj}^o\rvert,
    \ \max_{\lambda\in\sigma_{m{+}1}} \lvert\lambda\rvert
    \leq 4\|B\|,
\]
\[
    \left\| \hat{f}_j - f_j \right\|_2,
    \ \left\| \hat{f}_{j,k} - f_{j,k}\right\|_2
    \leq
    \frac4n \|B\|
         \max\left\{
         \frac{1}{
             \min\limits_{1\leq l{\neq}p \leq m }{|\lambda_l - \lambda_p|}},
         \frac{1}{
             \min\limits_{1\leq l \leq m}{|\lambda_l|}}
         \right\}
\]
for all \( j{=}\overline{1,m}, k{=}\overline{1,n-1} \).
\/}

\subsection*{Kronecker products}

Now we consider Kronecker products
\[
    A\otimes B =
    \begin{pmatrix}
        a_{11} B & \cdots & a_{1n} B \\
        \vdots   & \ddots & \vdots \\
        a_{n1} B & \cdots & a_{nn}
    \end{pmatrix}
    \in \mathbb{K}^{{mn}{\times}{mn}}
\]
of squared matrices
\( A={(a_{ij})}\in\mathbb{K}^{n{\times}n},
 \ B={(b_{ij})}\in\mathbb{K}^{m{\times}m} \)
and small-norm perturbations
\[
    A\otimes B - F.
\]

We will also address special
    perturbations of the following form (cf.~\cite{XIANG2005210}):
\[
    (A-\Delta A)\otimes (B - \Delta B).
\]

\section*{References}
\bibliographystyle{iopart-num}
\bibliography{sergey_kozlukov_jpcs}{}
\end{document}
